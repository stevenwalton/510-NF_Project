\documentclass{article} 
\usepackage[utf8]{inputenc}

\title{CIS 510 Computer Vision - Project Proposal:\\
	Faking COVID-19: Fooling diagnosis with Normalizing flows} 
\author{Guzman, Luis and Walton, Steven}
\date{April 2020}

\usepackage{natbib} 
\usepackage{graphicx} 
\usepackage{amsmath} 

\begin{document}
\maketitle

\section{Problem Description}
\label{intro}

Generative models such as Generative Adversarial Networks (GAN) and Variational
AutoEncoders (VAE) have recently attracted lots of attention due to  their
ability to generate novel data which were unseen by the model. For example, a
GAN that has been trained to learn what a cat looks like is able to generate a
new cat that has not been seen before\cite{tcdne}. Normalizing flows (NF) are
a class of Generative Models that learn the probability distribution function
(PDF) and can evaluate the log-likelihood at runtime. This alows NFs to sample
from the PDF and generate `new' data. This is useful when the generative
process is difficult. However, it is yet to be proved if generative machine
learning models can substitute real generative processes such as simulations.
In this work, our final objective is to train a generative model using
normalizing flows to produce Chest X-ray images with pneumonia that are able to
consistently fool a classifier.

The intent of this project is to learn more about Normalizing Flows and their
usage. Herein we describe a plan to learn more about these models so that we
may be able to use them within our own research, which is focused on Generative
Models. We recognize that this work in itself may not be an appropriate topic
for publication. Nonetheless, we believe that during the learning process we
will become more informed on the state-of-the-art uses and that this knowledge
will lead to more innovative ideas, thus allowing us to form better research
topics in the area. From our limited knowledge we believe that NF's ability to
learn the PDF and likelihood will play an important role in the future of
machine learning.

\subsection{Tasks}

\begin{enumerate} 
	\item \textbf{Literature Review}: Gaining an understanding of
		Normalizing Flows, their strengths, weaknesses and the
		situations where they can be useful. Literature review of
		relevant NF architectures NICE\cite{nice},
		RealNVP\cite{real_nvp}, FFJORD\cite{ffjord}, Glow\cite{glow}.  
	\item \textbf{Learn by Experimentation}: Experiment with model
		architecture or new bijective functions to include in the
		flows.  
	\item \textbf{Modeling With Standard Datasets}: Implement our NF model
		and test its performance by generating images from toy datasets
		like FashionMNIST and CIFAR.  
	\item \textbf{Develop Classifier}: Implement a pneumonia classifier
		using the Chest X-ray dataset (\textit{stretch goal}).  
	\item \textbf{Develop NF Pneumonia Model}: Use our NF model to generate
		fake X-ray images with pneumonia in an attempt to fool the
		classifier (\textit{stretch goal}).  \end{enumerate}

\section{Member roles} 
We attempt to break down these tasks so that there is an
even amount of work between partners. To begin we will first start with task 1,
literature review, which will continue throughout the project. For this, we
propose that each member brings one to two papers per week and we discuss what
we learned from these. We will provide slides as a summary for each paper. The
second task is to develop simple models of normalizing flows. In this we
propose that each member develop their own simplified normalizing flow program
and present it to one another. This will ensure that each member has a
practical understanding of NFs. Third, we propose to develop the model together
focusing on a standard dataset. We will combine our efforts from the previous
work and distribute the workload best when we know more about our individual
strengths. Finally, we propose that we perform tasks 4 and 5 in parallel,
leaning on one another's strengths. Throughout this process we will comment our
code and hold frequent meetings to discuss our choices and ideas, this ensures
that we help one another strengthen our respective weaknesses. 

\section{Resources and Tools}

\subsection{Datasets} 

\begin{itemize} 
	\item MNIST\footnote{http://yann.lecun.com/exdb/mnist/} and
		FashionMNIST\footnote{https://github.com/zalandoresearch/fashion-mnist} 
	\item CIFAR-10 and CIFAR-100\footnote{https://www.cs.toronto.edu/~kriz/cifar.html}
	\item Chest X-ray dataset
\footnote{https://www.kaggle.com/paultimothymooney/chest-xray-pneumonia}
\end{itemize} 

\subsection{Base Code}

We plan to implement the models in Python using the PyTorch deep learning
library.

We are also consulting the available code from the relevant papers in our
literature review: 
\begin{itemize} 
	\item NICE\footnote{https://github.com/laurent-dinh/nice} 
	\item Real-NVP\footnote{https://github.com/tensorflow/models/tree/master/research/real\_nvp}
	\item Glow\footnote{https://github.com/openai/glow} 
	\item FFJORD\footnote{https://github.com/lucasdurand/ffjord} 
\end{itemize}

We will use these codes to strengthen our knowledge of NFs and experiment with
different configurations. 

\section{Reservations} 
From our understanding, training NF-based models can be
time consuming and computationally expensive. Furthermore, it certainly can be
the case that the results we obtain (generated `fake' images) might not be good
enough to accomplish our final objective of fooling a pneumonia classifier.
Additionally, we are concerned about how much we will be able to accomplish
within the shortened time span of the quarter system, as neither of us have a
background in Normalizing Flows. 

\section{Relationship to background} 

\subsection{Luis} 
In Luis's case, he has some previous experience with PyTorch
and generative models in the form of GANs. However, Normalizing Flows are a
subject totally new to him that will require considerable literature review and
effort. From this project, Luis expects to get some insight into the intuition
and inner workings of normalizing flows and, hopefully, come out with an idea
that could be applied to his own research.

\subsection{Steven} 
Steven has previous experience with PyTorch and generative
models like GANs. He has done some minor literature review on the subject but
does not feel he has a full grasp on the models. From this project, Steven
seeks to gain a deeper understanding of Normalizing Flows and to better
understand how he can apply them to his own research. He believes that Flow
based models will be helpful within scientific applications of machine
learning. 

\bibliographystyle{abbrv} 
\bibliography{references} 
\end{document}

